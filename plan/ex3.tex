\documentclass{TIJMUjiaoanSY}
\pagestyle{empty}

\begin{document}

\kecheng{分子生物计算}
\shiyan{实验3 \ 生物序列处理}
\jiaoshi{伊现富}
\zhicheng{讲师}
\riqi{2018年11月27日8:00-9:40}
\duixiang{生物医学工程与技术学院2016级生信班(本)}
\renshu{28}
\leixing{验证型}
\fenzu{一人一机}
\xueshi{2}
\jiaocai{Perl语言在生物信息学中的应用——基础篇}

\firstHeader
\maketitle
\thispagestyle{empty}

\mudi{
\begin{itemize}
  \item 了解Perl语言中的标量和数组;Perl语言中的上下文。
  \item 熟悉Perl语言中的字符串操作;读取文件中数据的方法。
  \item 掌握Perl语言在生物序列数据处理中的应用。
\end{itemize}
}

\fenpei{
\begin{itemize}
  \item (5')变量:比较Perl语言中的标量和数组。
  \item (5')字符串操作:总结Perl语言中常见的字符串操作。
  \item (5')读取文件:总结在Perl语言中读取文件数据的基本步骤。
  \item (5')上下文:比较Perl语言中的标量上下文和列表上下文。
  \item (80')实验操作:应用Perl语言处理生物序列数据。
\end{itemize}
}

\cailiao{
\begin{itemize}
  \item 主要仪器:一台安装有Perl语言(Linux操作系统)的计算机。
\end{itemize}
}
\zhongdian{
\begin{itemize}
  \item 重点难点:应用Perl语言处理生物序列数据。
  \item 解决策略:通过演示进行学习,通过练习熟练掌握。
\end{itemize}
}

\sikao{
\begin{itemize}
  \item 比较Perl语言中的标量和数组。
  \item 总结Perl语言中常见的字符串操作。
  \item 总结Perl语言中读取文件的基本步骤。
  \item 比较Perl语言中的标量上下文和列表上下文。
\end{itemize}
}

\cankao{
\begin{itemize}
  \item Beginning Perl for Bioinformatics, James Tisdall, O'Reilly Media, 2001.
  \item Perl语言入门(第六版),Randal L. Schwartz, brian d foy \& Tom Phoenix著,盛春\ 译,东南大学出版社,2012。
  \item Mastering Perl for Bioinformatics, James Tisdall, O'Reilly Media, 2003.
  \item 维基百科等网络资源。
\end{itemize}
}

\firstTail

\newpage
\otherHeader

\begin{enumerate}
\vspace{-1em}
\begin{multicols}{2}
  \item 变量(5分钟)
    \begin{enumerate}
      \item 标量:scalar,单数,\verb|$|
      \item 数组:array,复数,\verb|@|;其中的元素是标量
    \end{enumerate}
  \item 字符串操作(5分钟)
    \begin{enumerate}
      \item 连接:\verb|.|等多种方法
      \item 替换:\verb|s///;|
      \item 反转:reverse
      \item 翻译:tr
    \end{enumerate}
  \item 读取文件(5分钟)
    \begin{enumerate}
      \item 关联文件和文件句柄
      \item 通过文件句柄读取数据
      \item 解关联文件和文件句柄
    \end{enumerate}
  \item 上下文(5分钟)
    \begin{enumerate}
      \item 标量上下文:返回标量
      \item 列表上下文:返回列表
    \end{enumerate}
\end{multicols}
\vspace{-1em}
  \item 实验操作(80分钟)
    \begin{enumerate}
      \item 存储并打印DNA序列
\begin{verbatim}
#!/usr/bin/perl -w

$DNA = 'ACGGGAGGACGGGAAAATTACTACGGCATTAGC';
print $DNA;

exit;
\end{verbatim}
      \item 连接DNA片段
\begin{verbatim}
#!/usr/bin/perl -w

$DNA1 = 'ACGGGAGGACGGGAAAATTACTACGGCATTAGC';
$DNA2 = 'ATAGTGCCGTGAGAGTGATGTAGTA';
print "Here are the original two DNA fragments:\n\n";
print $DNA1, "\n";
print $DNA2, "\n\n";

$DNA3 = "$DNA1$DNA2";
print "Concatenation of the first two fragments (version 1):\n\n";
print "$DNA3\n\n";

$DNA3 = $DNA1 . $DNA2;
print "Concatenation of the first two fragments (version 2):\n\n";
print "$DNA3\n\n";

print "Concatenation of the first two fragments (version 3):\n\n";
print $DNA1, $DNA2, "\n";
\end{verbatim}
      \item 把DNA转录成RNA
\begin{verbatim}
#!/usr/bin/perl -w

$DNA = 'ACGGGAGGACGGGAAAATTACTACGGCATTAGC';
print "Here is the starting DNA:\n\n";
print "$DNA\n\n";

$RNA = $DNA;
$RNA =~ s/T/U/g;
print "Here is the result of transcribing the DNA to RNA:\n\n";
print "$RNA\n";
\end{verbatim}

\otherTail
\newpage
\otherHeader

      \item 获取DNA序列的反向互补序列
\begin{verbatim}
#!/usr/bin/perl -w

$DNA = 'ACGGGAGGACGGGAAAATTACTACGGCATTAGC';
print "Here is the starting DNA:\n\n";
print "$DNA\n\n";

$revcom = reverse $DNA;
$revcom =~ tr/ACGTacgt/TGCAtgca/;
print "Here is the reverse complement DNA:\n\n";
print "$revcom\n";
\end{verbatim}
      \item 从文件中读取蛋白质序列
\begin{verbatim}
#!/usr/bin/perl -w

$proteinfilename = 'NM_021964fragment.pep';

open( PROTEINFILE, $proteinfilename );
@protein = <PROTEINFILE>;
print @protein;
close PROTEINFILE;
\end{verbatim}
    \end{enumerate}
\end{enumerate}

\otherTail


\end{document}
