\documentclass{TIJMUjiaoanLL}
\pagestyle{empty}

\begin{document}

\kecheng{分子生物计算}
\neirong{基序和循环 \ / 第5章}
\jiaoshi{伊现富}
\zhicheng{讲师}
\riqi{2018年11月28\&3日15:30-17:10\&10:00-11:40}
\duixiang{生物医学工程与技术学院2016级生信班(本)}
\renshu{28}
\fangshi{理论讲授}
\xueshi{4}
\jiaocai{Perl语言在生物信息学中的应用——基础篇}

\firstHeader
\maketitle
\thispagestyle{empty}

\mudi{
\begin{itemize}
  \item 掌握:流程控制的语句及语法;文件交互的方法;join、split、substr和tr等函数的基本用法;Perl语言中真与假的判断。
  \item 熟悉:正则表达式的基本用法;获取键盘输入的方法。
  \item 了解:常见的特殊变量和文件测试;变量初始化的方法。
  \item 自学:join、split和substr等函数的高级用法。
\end{itemize}
}

\fenpei{
\begin{itemize}
  \item (5')引言与导入:回顾上一章学习的知识点,介绍本章将要学习的内容。
  \item (35')流程控制:介绍流程控制的概念和指令,讲解流程控制的分类及各自的语法。
  \item (10')代码布局:比较不同的代码布局。
  \item (50')查找基序:通过在蛋白质序列中查找用户指定基序的Perl程序讲解Perl语言的相关知识。
  \item (70')计数核苷酸:通过计数核苷酸的Perl程序讲解字符串操作等Perl语言的相关知识。
  \item (25')写入文件:通过保存核苷酸计数结果的Perl程序讲解写入文件的方法。
  \item (5')总结与答疑:总结授课内容中的知识点与技能,解答学生疑问。
\end{itemize}
}

\zhongdian{
\begin{itemize}
  \item 重点:流程控制的语句及语法;常用字符串操作函数的用法;真与假的判断。
  \item 难点:Perl语言中真与假的判断。
  \item 解决策略:通过实例演示帮助学生理解、记忆。
\end{itemize}
}

\waiyu{
\vspace*{-10pt}
\begin{multicols}{2}
流程控制(flow control)

真(true)

假(false)

基序(motif)

初始化(initialization)

声明(declare)
\end{multicols}
\vspace*{-10pt}
}

\fuzhu{
\begin{itemize}
  \item 多媒体:流程控制的各种语法;正则表达式的使用。
  \item 板书:真与假的判断法则;正则表达式解析。
  \item 演示:常见字符串处理函数的使用。
\end{itemize}
}

\sikao{
\vspace*{-10pt}
\begin{multicols}{2}
\begin{itemize}
  \item 总结Perl语言中流程控制的语句及其语法。
  \item 总结Perl语言真与假的判断法则。
  \item 总结字符串和数组互转的方法。
  \item 总结正则表达式和模式匹配的使用。
  \item 总结Perl对数字和字符串的智能化处理。
  \item 总结substr和tr处理字符串的用法。
\end{itemize}
\end{multicols}
\vspace*{-10pt}
}

\cankao{
\begin{itemize}
  \item Beginning Perl for Bioinformatics, James Tisdall, O'Reilly Media, 2001.
  \item Perl语言入门(第六版),Randal L. Schwartz, brian d foy \& Tom Phoenix著,盛春\ 译,东南大学出版社,2012。
  \item Mastering Perl for Bioinformatics, James Tisdall, O'Reilly Media, 2003.
  \item 维基百科等网络资源。
\end{itemize}
}

\firstTail

\newpage
\otherHeader

\begin{enumerate}
  \item 引言与导入(5分钟)
    \begin{enumerate}
      \item 已经学习
	\begin{itemize}
	  \item Perl语言基础:标量、数组、字符串操作、读取文件
	  \item 生物序列处理:拼接DNA片段、DNA转录成RNA、获取反向互补序列
	\end{itemize}
      \item 即将学习
	\begin{itemize}
	  \item Perl语言基础:条件测试、循环、正则表达式、写入文件
	  \item 生物序列处理:查找基序、计数核苷酸
	\end{itemize}
    \end{enumerate}
  \item 流程控制(35分钟)
    \begin{enumerate}
      \item 简介
	\begin{enumerate}
	  \item 定义:在程序运行时,个别的指令(或是陈述、子程序)运行或求值的顺序
	  \item 指令
	    \begin{itemize}
	      \item 继续运行位于不同位置的一段指令
	      \item 若特定条件成立时,运行一段指令
	      \item 运行一段指令若干次,直到特定条件成立为止
	      \item 运行位于不同位置的一段指令,但完成后会继续运行原来要运行的指令
	      \item 停止程序,不运行任何指令
	    \end{itemize}
	\end{enumerate}
      \item \textcolor{red}{【重点】}分类
	\begin{enumerate}
	  \item 默认:除非明确指明不按顺序执行,否则程序将从最顶端的第一个语句开始,顺序执行到最底端的最后一个语句
	  \item 条件判断:只在条件测试成功的前提下执行相应的语句,否则直接跳过这些语句;if、if-else、unless\textcolor{red}{(通过实例详细讲解各自的语法,比较常规写法和简写,比较if和unless)}
	    \begin{itemize}
	      \item Perl程序5.1:使用if-elsif-else
	    \end{itemize}
\vspace{-1em}
\begin{multicols}{2}
\begin{verbatim}
if( 1 == 1 ) {
print "1 equals 1\n";
}
if( 1 ) {
print "1 evaluates to true\n";
}
\end{verbatim}
\begin{verbatim}
if( 1 == 0 ) {
print "1 equals 0\n";
}
if( 0 ) {
print "0 evaluates to true\n";
}
\end{verbatim}
\end{multicols}
\vspace{-1em}
	  \item 循环:一直重复语句,直到相应的测试失败为止;while、for、foreach
	    \begin{itemize}
	      \item Perl程序5.2:使用while从文件中读取蛋白质序列数据
	    \end{itemize}
	\end{enumerate}
      \item \textcolor{red}{【重点、难点】}真和假
	\begin{enumerate}
	  \item 数字:0为假,其他为真
	  \item 字符串:空字符串和字符串\verb|'0'|\textcolor{red}{(唯一为假的非空字符串)}为假,其他为真
	  \item 其他:先转换成数字或字符串再行判断
	\end{enumerate}
    \end{enumerate}
  \item 代码布局(10分钟)\textcolor{red}{(比较不同代码布局的优缺点)}
\vspace{-1em}
\begin{multicols}{2}
\begin{verbatim}
while ( $alive ) {
  if ( $needs_nutrients ) {
    print "Needs nutrients\n";
  }
}


\end{verbatim}
\begin{verbatim}
while ( $alive )
{
  if ( $needs_nutrients )
  {
    print "Needs nutrients\n";
  }
}
\end{verbatim}
\end{multicols}
\vspace{-1em}

\otherTail
\newpage
\otherHeader

  \item 查找基序(50分钟)
    \begin{enumerate}
      \item 问题分析与解决策略:基序长度不定有变体;正则表达式
      \item Perl程序5.3:在蛋白质序列中查找用户指定的基序
      \item 获取键盘输入:\verb|<STDIN>|,chomp vs. chop
      \item \textcolor{red}{【重点】}数组变标量:\verb|$protein = join ( '', @protein);|\textcolor{red}{(拼接两个DNA片段的又一种方法)}
      \item do-until:先执行后测试\textcolor{red}{(与until进行比较)}
      \item 正则表达式:\verb|$protein =~ s/\s//g;|,\verb|$motif =~ /^\s*$/|\textcolor{red}{(结合实例解释其中的字符集、元字符、量词、锚位等)}
      \item 模式匹配:\verb|$protein =~ /$motif/|\textcolor{red}{(比较变量内插和直接使用字符串的优缺点)}
    \end{enumerate}
  \item 计数核苷酸(70分钟)\textcolor{red}{(同样的碱基计数,不同的策略方法,各有优势与劣势)}
    \begin{enumerate}
      \item 伪代码\textcolor{red}{(通过简洁的伪代码理清思路)}
      \item 策略\textcolor{red}{(思考每种策略的优缺点)}
	\begin{enumerate}
	  \item 把DNA拆解成单个碱基,存储到数组中,对数组中的元素进行迭代处理
	  \item 对DNA字符串中的位置进行迭代处理
	\end{enumerate}
      \item 把字符串拆解成数组
	\begin{enumerate}
	  \item Perl程序5.4:对DNA序列中的碱基进行计数
	  \item \textcolor{red}{【重点】}split:\verb|@DNA = split( '', $DNA );|\textcolor{red}{(和join进行比较)}
	  \item 初始化
	    \begin{itemize}
	      \item 未初始化变量的值为\verb|'undef'|——0或者空字符串\textcolor{red}{(Perl中的上下文无处不在)}
	      \item Perl程序5.5:Perl对数字和字符串的智能化处理
	    \end{itemize}
	  \item foreach:\verb|foreach $base (@DNA) {|\textcolor{red}{(比较自定义变量和内置变量的利弊)}
	  \item +1:至少四种实现方法\textcolor{red}{(条条大路通罗马,比较各种方法的优缺点)}
	\end{enumerate}
      \item 操作字符串
	\begin{enumerate}
	  \item Perl程序5.6:对DNA序列中的碱基进行计数
	  \item 文件测试:unless,-e
	  \item for vs. foreach\textcolor{red}{(语法不同,本质一样)}
	  \item 索引:不管是字符串还是数组元素,都从0开始索引
    \item \textcolor{red}{【重点】}substr:\verb|$base = substr($DNA, $position, 1);|\textcolor{red}{(通过实例详细讲解其用法;substr vs. splice——操作字符串 vs. 操作数组)}
	\end{enumerate}
    \end{enumerate}
  \item 写入文件(25分钟)
    \begin{enumerate}
      \item Perl程序5.7:对DNA序列中的碱基进行计数,结果保存到文件
      \item 读取文件 vs. 写入文件
      \item while:妙用正则匹配进行计数
      \item \textcolor{red}{【重点】}tr:妙用tr进行计数,优劣并存
    \end{enumerate}
  \item 总结与答疑(5分钟)
    \begin{enumerate}
      \item 知识点
	\vspace{-1em}
	\begin{multicols}{2}
	\begin{itemize}
	  \item 流程控制:条件、循环
	  \item 文件交互:打开、读取、写入
	  \item 正则匹配:正则表达式,模式匹配
	  \item 字符串操作:join、split、substr、tr
	  \item 其他:真与假、获取键盘输入、变量递增、文件测试、……
	\end{itemize}
      \end{multicols}
	\vspace{-1em}
      \item 技能
	\begin{itemize}
	  \item 熟练使用Perl语言中的各种流程控制语句
	  \item 能够编写在DNA或者蛋白质序列中查找基序的程序
	  \item 能编写对DNA序列中的核苷酸进行计数的程序
	\end{itemize}
    \end{enumerate}
\end{enumerate}

\otherTail


\end{document}
