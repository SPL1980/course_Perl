\documentclass{TIJMUjiaoanSY}
\pagestyle{empty}

\begin{document}

\kecheng{分子生物计算}
\shiyan{实验8 \ 翻译DNA序列}
\jiaoshi{伊现富}
\zhicheng{讲师}
\riqi{2019年1月1日8:00-9:40}
\duixiang{生物医学工程与技术学院2016级生信班(本)}
\renshu{28}
\leixing{验证型}
\fenzu{一人一机}
\xueshi{2}
\jiaocai{Perl语言在生物信息学中的应用——基础篇}

\firstHeader
\maketitle
\thispagestyle{empty}

\mudi{
\begin{itemize}
  \item 了解阅读框的概念。
  \item 熟悉FASTA格式。
  \item 掌握:散列的使用;子程序的应用。
\end{itemize}
}

\fenpei{
\begin{itemize}
  \item (100')实验操作:编写Perl程序把FASTA文件中的DNA序列翻译成蛋白质。
\end{itemize}
}

\cailiao{
\begin{itemize}
  \item 主要仪器:一台安装有Perl语言(Linux操作系统)的计算机。
\end{itemize}
}
\zhongdian{
\begin{itemize}
  \item 重点难点:散列的使用。
  \item 解决策略:通过演示进行学习,通过练习熟练掌握。
\end{itemize}
}

\sikao{
\begin{itemize}
  \item 如何使用正则表达式表征密码子?
  \item 比较Perl语言中常见的三种数据类型。
  \item 举例说明常见的序列格式。
  \item 举例说明6种阅读框。
\end{itemize}
}

\cankao{
\begin{itemize}
  \item Beginning Perl for Bioinformatics, James Tisdall, O'Reilly Media, 2001.
  \item Perl语言入门(第六版),Randal L. Schwartz, brian d foy \& Tom Phoenix著,盛春\ 译,东南大学出版社,2012。
  \item Mastering Perl for Bioinformatics, James Tisdall, O'Reilly Media, 2003.
  \item 维基百科等网络资源。
\end{itemize}
}

\firstTail

\newpage
\otherHeader

\begin{enumerate}
  \item 实验操作(100分钟)
    \begin{enumerate}
      \item 相关的子程序
	\begin{itemize}
	  \item codon2aa:把密码子翻译成氨基酸
	  \item dna2peptide:把DNA翻译成多肽
	  \item get\_file\_data,extract\_sequence\_from\_fasta\_data:从FASTA文件中提取序列数据
	  \item print\_sequence:输出格式化的序列数据
	  \item revcom:获取反向互补序列
	  \item translate\_frame:翻译DNA序列的一个阅读框
	\end{itemize}
      \item 把DNA序列翻译成蛋白质
\begin{verbatim}
#!/usr/bin/perl

use strict; use warnings; use BeginPerlBioinfo;

my $dna     = 'CGACGTCTTCGTACGGGACTAGCTCGTGTCGGTCGC';
my $protein = '';
my $codon;

for ( my $i = 0 ; $i < ( length($dna) - 2 ) ; $i += 3 ) {
    $codon = substr( $dna, $i, 3 );
    $protein .= codon2aa($codon);
}
print "I translated the DNA\n\n$dna\n\n  into the protein\n\n$protein\n\n";
\end{verbatim}
      \item 读取FASTA文件并提取其中的序列数据
\begin{verbatim}
#!/usr/bin/perl

use strict; use warnings; use BeginPerlBioinfo;

my @file_data = ();
my $dna       = '';

@file_data = get_file_data("sample.dna");
$dna = extract_sequence_from_fasta_data(@file_data);
print_sequence( $dna, 25 );
\end{verbatim}
      \item 把FASTA文件中的DNA翻译成蛋白质并进行格式化输出
\begin{verbatim}
#!/usr/bin/perl

use strict; use warnings; use BeginPerlBioinfo;

my @file_data = ();
my $dna       = '';
my $protein   = '';

@file_data = get_file_data("sample.dna");
$dna = extract_sequence_from_fasta_data(@file_data);
$protein = dna2peptide($dna);
print_sequence( $protein, 25 );
\end{verbatim}

\otherTail
\newpage
\otherHeader

      \item 从六个阅读框上翻译DNA序列
\begin{verbatim}
#!/usr/bin/perl

use strict; use warnings; use BeginPerlBioinfo;

my @file_data = ();
my $dna       = '';
my $revcom    = '';
my $protein   = '';

@file_data = get_file_data("sample.dna");
$dna = extract_sequence_from_fasta_data(@file_data);

print "\n -------Reading Frame 1--------\n\n";
$protein = translate_frame( $dna, 1 );
print_sequence( $protein, 70 );

print "\n -------Reading Frame 2--------\n\n";
$protein = translate_frame( $dna, 2 );
print_sequence( $protein, 70 );

print "\n -------Reading Frame 3--------\n\n";
$protein = translate_frame( $dna, 3 );
print_sequence( $protein, 70 );

$revcom = revcom($dna);

print "\n -------Reading Frame 4--------\n\n";
$protein = translate_frame( $revcom, 1 );
print_sequence( $protein, 70 );

print "\n -------Reading Frame 5--------\n\n";
$protein = translate_frame( $revcom, 2 );
print_sequence( $protein, 70 );

print "\n -------Reading Frame 6--------\n\n";
$protein = translate_frame( $revcom, 3 );
print_sequence( $protein, 70 );
\end{verbatim}
    \end{enumerate}
\end{enumerate}

\otherTail

%\parpic[fr]{\includegraphics[width=\textwidth]{}}

\end{document}
