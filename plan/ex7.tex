\documentclass{TIJMUjiaoanSY}
\pagestyle{empty}

\begin{document}

\kecheng{分子生物计算}
\shiyan{实验7 \ 生成随机DNA并进行分析}
\jiaoshi{伊现富}
\zhicheng{讲师}
\riqi{2018年12月26日8:00-9:40}
\duixiang{生物医学工程与技术学院2016级生信班(本)}
\renshu{28}
\leixing{验证型}
\fenzu{一人一机}
\xueshi{2}
\jiaocai{Perl语言在生物信息学中的应用——基础篇}

\firstHeader
\maketitle
\thispagestyle{empty}

\mudi{
\begin{itemize}
  \item 了解伪代码和子程序的设计与编写。
  \item 熟悉自上而下和自下而上的程序设计理念。
  \item 掌握Perl语言中的嵌套循环。
\end{itemize}
}

\fenpei{
\begin{itemize}
  \item (90)实验操作:编写Perl程序生成随机DNA序列并计算它们的相似性百分比。
\end{itemize}
}

\cailiao{
\begin{itemize}
  \item 主要仪器:一台安装有Perl语言(Linux操作系统)的计算机。
\end{itemize}
}
\zhongdian{
\begin{itemize}
  \item 重点难点:嵌套循环。
  \item 解决策略:通过演示进行学习,通过练习熟练掌握。
\end{itemize}
}

\sikao{
\begin{itemize}
  \item 比较自上而下和自下而上的两种程序设计理念。
  \item 解释嵌套循环的逻辑步骤。
\end{itemize}
}

\cankao{
\begin{itemize}
  \item Beginning Perl for Bioinformatics, James Tisdall, O'Reilly Media, 2001.
  \item Perl语言入门(第六版),Randal L. Schwartz, brian d foy \& Tom Phoenix著,盛春\ 译,东南大学出版社,2012。
  \item Mastering Perl for Bioinformatics, James Tisdall, O'Reilly Media, 2003.
  \item 维基百科等网络资源。
\end{itemize}
}

\firstTail

\newpage
\otherHeader

\begin{enumerate}
  \item 实验操作(90分钟)
    \begin{enumerate}
      \item 生成一系列长短不一的随机DNA序列片段
\begin{verbatim}
#!/usr/bin/perl
use strict; use warnings;
my $maximum_length = 30; my $minimum_length = 15;
my $size_of_set = 12; my @random_DNA = ();
srand( time | $$ );
@random_DNA = make_random_DNA_set( $minimum_length, $maximum_length, $size_of_set );
print "Here is an array of $size_of_set randomly generated DNA sequences\n";
print "  with lengths between $minimum_length and $maximum_length:\n\n";
foreach my $dna (@random_DNA) {
    print "$dna\n";
}
print "\n";

sub make_random_DNA_set {
    my ( $minimum_length, $maximum_length, $size_of_set ) = @_;
    my $length;
    my $dna;
    my @set;
    for ( my $i = 0 ; $i < $size_of_set ; ++$i ) {
        $length = randomlength( $minimum_length, $maximum_length );
        $dna = make_random_DNA($length);
        push( @set, $dna );
    }
    return @set;
}
sub randomlength {
    my ( $minlength, $maxlength ) = @_;
    return ( int( rand( $maxlength - $minlength + 1 ) ) + $minlength );
}
sub make_random_DNA {
    my ($length) = @_;
    my $dna;
    for ( my $i = 0 ; $i < $length ; ++$i ) {
        $dna .= randomnucleotide();
    }
    return $dna;
}
sub randomnucleotide {
    my (@nucleotides) = ( 'A', 'C', 'G', 'T' );
    return randomelement(@nucleotides);
}
sub randomelement {
    my (@array) = @_;
    return $array[ rand @array ];
}
\end{verbatim}

\otherTail
\newpage
\otherHeader

      \item 计算随机DNA的相似性百分比
\begin{verbatim}
#!/usr/bin/perl
use strict; use warnings;
my $percent; my @percentages; my $result; my @random_DNA = ();
srand( time | $$ );
@random_DNA = make_random_DNA_set( 10, 10, 10 );
for ( my $k = 0 ; $k < scalar @random_DNA - 1 ; ++$k ) {
    for ( my $i = ( $k + 1 ) ; $i < scalar @random_DNA ; ++$i ) {
        $percent = matching_percentage( $random_DNA[$k], $random_DNA[$i] );
        push( @percentages, $percent );
} }
$result = 0;
foreach $percent (@percentages) { $result += $percent; }
$result = $result / scalar(@percentages); $result = int( $result * 100 );
print "In this run of the experiment, the average percentage of \n";
print "matching positions is $result%\n\n";
sub matching_percentage {
    my ( $string1, $string2 ) = @_;
    my ($length) = length($string1); my ($position); my ($count) = 0;
    for ( $position = 0 ; $position < $length ; ++$position ) {
        if ( substr( $string1, $position, 1 ) eq substr( $string2, $position, 1 ))
        { ++$count; }
    }
    return $count / $length;
}
sub make_random_DNA_set {
    my ( $minimum_length, $maximum_length, $size_of_set ) = @_;
    my $length; my $dna; my @set;
    for ( my $i = 0 ; $i < $size_of_set ; ++$i ) {
        $length = randomlength( $minimum_length, $maximum_length );
        $dna = make_random_DNA($length); push( @set, $dna );
    }
    return @set;
}
sub randomlength {
    my ( $minlength, $maxlength ) = @_;
    return ( int( rand( $maxlength - $minlength + 1 ) ) + $minlength );
}
sub make_random_DNA {
    my ($length) = @_; my $dna;
    for ( my $i = 0 ; $i < $length ; ++$i ) {
        $dna .= randomnucleotide(); }
    return $dna;
}
sub randomnucleotide {
    my (@nucleotides) = ( 'A', 'C', 'G', 'T' );
    return randomelement(@nucleotides);
}
sub randomelement { my (@array) = @_; return $array[ rand @array ]; }
\end{verbatim}
    \end{enumerate}
\end{enumerate}

\otherTail

\end{document}
