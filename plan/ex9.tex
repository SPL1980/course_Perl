\documentclass{TIJMUjiaoanSY}
\pagestyle{empty}

\begin{document}

\kecheng{分子生物计算}
\shiyan{实验9 \ 制作酶切图谱}
\jiaoshi{伊现富}
\zhicheng{讲师}
\riqi{2019年10月30日13:30-15:10}
\duixiang{生物医学工程与技术学院2017级生信班(本)}
\renshu{28}
\leixing{验证型}
\fenzu{一人一机}
\xueshi{2}
\jiaocai{Perl语言在生物信息学中的应用——基础篇}

\firstHeader
\maketitle
\thispagestyle{empty}

\mudi{
\begin{itemize}
  \item 了解REBASE数据库中bionet格式。
  \item 熟悉模式匹配设计的特殊变量。
  \item 掌握正则表达式在酶切图谱制作中的应用。
\end{itemize}
}

\fenpei{
\begin{itemize}
  \item (90')实验操作:编写Perl程序制作酶切图谱。
\end{itemize}
}

\cailiao{
\begin{itemize}
  \item 主要仪器:一台安装有Perl语言(Linux操作系统)的计算机。
\end{itemize}
}
\zhongdian{
\begin{itemize}
  \item 重点难点:正则表达式的应用。
  \item 解决策略:通过演示进行学习,通过练习熟练掌握。
\end{itemize}
}

\sikao{
\begin{itemize}
  \item 举例说明正则表达式中的元字符
  \item 举例说明模式匹配中涉及的特殊变量。
  \item 解释逻辑操作符的求值顺序。
\end{itemize}
}

\cankao{
\begin{itemize}
  \item Beginning Perl for Bioinformatics, James Tisdall, O'Reilly Media, 2001.
  \item Perl语言入门(第六版),Randal L. Schwartz, brian d foy \& Tom Phoenix著,盛春\ 译,东南大学出版社,2012。
  \item Mastering Perl for Bioinformatics, James Tisdall, O'Reilly Media, 2003.
  \item 维基百科等网络资源。
\end{itemize}
}

\firstTail

\newpage
\otherHeader

\begin{enumerate}
  \item 实验操作(90分钟)
    \begin{enumerate}
      \item 把IUB代码翻译成正则表达式
\begin{verbatim}
sub IUB_to_regexp {
    my ($iub) = @_;
    my $regular_expression = '';
    my %iub2character_class = (
        A => 'A',
        C => 'C',
        G => 'G',
        T => 'T',
        R => '[GA]',
        Y => '[CT]',
        M => '[AC]',
        K => '[GT]',
        S => '[GC]',
        W => '[AT]',
        B => '[CGT]',
        D => '[AGT]',
        H => '[ACT]',
        V => '[ACG]',
        N => '[ACGT]',
    );
    $iub =~ s/\^//g;
    for ( my $i = 0 ; $i < length($iub) ; ++$i ) {
        $regular_expression .= $iub2character_class{ substr( $iub, $i, 1 ) };
    }
    return $regular_expression;
}
\end{verbatim}
      \item 解析REBASE中的bionet数据文件
\begin{verbatim}
sub parseREBASE {
    my ($rebasefile) = @_;
    use strict; use warnings; use BeginPerlBioinfo;
    my @rebasefile  = ();
    my %rebase_hash = ();
    my $name; my $site; my $regexp;
    my $rebase_filehandle = open_file($rebasefile);
    while (<$rebase_filehandle>) {
        ( 1 .. /Rich Roberts/ ) and next;
        /^\s*$/ and next;
        my @fields = split( " ", $_ );
        $name = shift @fields;
        $site = pop @fields;
        $regexp = IUB_to_regexp($site);
        $rebase_hash{$name} = "$site $regexp";
    }
    return %rebase_hash;
}
\end{verbatim}

\otherTail
\newpage
\otherHeader

      \item 根据用户的查询制作酶切图谱
\begin{verbatim}
#!/usr/bin/perl
use strict; use warnings;
use BeginPerlBioinfo;
my %rebase_hash      = ();
my @file_data        = ();
my $query            = '';
my $dna              = '';
my $recognition_site = '';
my $regexp           = '';
my @locations        = ();
@file_data = get_file_data("sample.dna");
$dna = extract_sequence_from_fasta_data(@file_data);
%rebase_hash = parseREBASE('bionet');
do {
    print "Search for what restriction site for (or quit)?: ";
    $query = <STDIN>;
    chomp $query;
    if ( $query =~ /^\s*$/ ) {
        exit;
    }
    if ( exists $rebase_hash{$query} ) {
        ( $recognition_site, $regexp ) = split( " ", $rebase_hash{$query} );
        @locations = match_positions( $regexp, $dna );
        if (@locations) {
            print "Searching for $query $recognition_site $regexp\n";
            print "A restriction site for $query at locations:\n";
            print join( " ", @locations ), "\n";
        }
        else {
            print "A restriction enzyme $query is not in the DNA:\n";
        }
    }
    print "\n";
} until ( $query =~ /quit/ );
exit;
sub match_positions {
    my ( $regexp, $sequence ) = @_;
    use strict;
    use BeginPerlBioinfo;    # see Chapter 6 about this module
    my @positions = ();
    while ( $sequence =~ /$regexp/ig ) {
        push( @positions, pos($sequence) - length($&) + 1 );
    }
    return @positions;
}
\end{verbatim}
    \end{enumerate}
\end{enumerate}

\otherTail

%\parpic[fr]{\includegraphics[width=\textwidth]{}}

\end{document}
